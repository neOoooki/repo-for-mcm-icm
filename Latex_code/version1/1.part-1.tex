\section{Introduction}


\subsection{Problem Background}







\subsection{Restatement of the Problem}

\noindent Considering the background information and restricted conditions identified in the problem statement, we need to solve the following problems:
\begin{itemize}[label=\Large\textbullet,leftmargin=*,itemsep=0.6ex,topsep=0.5ex]
    \item Develop a model to describe the frequency of stair usage.
    \item Create a model to determine whether a certain direction of travel was favored.
\end{itemize}

\begin{itemize}[label=\Large\textbullet,leftmargin=*,itemsep=0.6ex,topsep=0.5ex]
    \item Develop a model to determine whether people traveled single file or side by side.
    \item Build a model to evaluate whether the observed wear matches historical and environmental factors.
    \item Create a model to estimate the age of the stairwell and its reliability.
    \item Design a model to identify signs of repairs or renovations.
    \item Determine whether the wear is consistent with materials from specific sources.
    \item Estimate typical usage and identify whether heavy short-term or prolonged lighter usage fits the observed patterns.
\end{itemize}







\subsection{Our Work}

\noindent In this paper, we carried out the following work to study stair usage frequency, service life, current condition, and pedestrian behavior patterns when using the stairs.
\begin{itemize}[label=\Large\ding{226},leftmargin=*,itemsep=0.6ex,topsep=0.5ex]
    \item \textbf{Task 1: Model Preparation}\\
    We determined material and environmental conditions of the target stairs, applied computer vision to obtain and process data, and fitted the wear depth distribution.
    
    \item \textbf{Task 2: Stair Wear Analysis Model}\\
    We modeled pedestrian traffic to characterize usage frequency, inferred preferred walking direction via changes in wear distribution, and computed pedestrian density from traffic flow to judge single-file or side-by-side walking.
    
    \item \textbf{Task 3:}\\
    \begin{itemize}[label=\Large\textbullet,leftmargin=*,itemsep=0.6ex,topsep=0.5ex]
        \item Analyze the relation among stair age, wear distribution, and pedestrian traffic.
        \item Compute theoretical wear from traffic and compare with measured wear; assess error.
        \item Estimate stair age from wear–traffic relationships and evaluate reliability.
        \item Identify signs of repairs or renovations via coefficient changes and verification.
        \item Check material consistency with presumed sources and verify with features.
        \item Estimate typical usage; distinguish heavy short-term vs. prolonged lighter usage.
    \end{itemize}
\end{itemize}
\section{Assumptions and Justifications}

\begin{itemize}[label=\Large\textbullet,leftmargin=*,itemsep=0.6ex,topsep=0.5ex]
    \item \textbf{Assumption 1: 示例标题}
    
    \textbf{Justification:} 示例文本。
    
    \item \textbf{Assumption 2: 示例标题}
    
    \textbf{Justification:} 示例文本。
    
    \item \textbf{Assumption 3: 示例标题}
    
    \textbf{Justification:} 示例文本。
    
    \item \textbf{Assumption 4: 示例标题}
    
    \textbf{Justification:} 示例文本。
\end{itemize}







\newpage
\section{Notations}
	The key mathematical notations used in this paper are listed in Table \ref{tab:notations}.

% 表格开始
\begin{table}[H]
	\centering
	% 设置标题格式:加粗
	\captionsetup{font=bf} 
	\caption{Notations used in this paper}
	\label{tab:notations}
	
	% 增加行高,使表格看起来更舒展(类似图片效果)
	\renewcommand{\arraystretch}{1.4}
	
	% 定义列格式:|c|c|c| 表示三列居中,且有竖线分隔
	\begin{tabular}{|c|c|c|}
		% 顶部粗线 (自定义粗细为1.5pt)
		\noalign{\hrule height 1.5pt}
		
		% 表头行:灰色背景,加粗斜体文字
		\rowcolor{gray!30} 
		\textit{\textbf{Symbols}} & \textit{\textbf{Definition}} & \textit{\textbf{Units}} \\
		
		\hline
		
		% 内容行
		$d(x,y)$ & The wear depth at each location $(x, y)$ & $m$ \\
		
		$N$ & Pedestrian flow & $people/s$ \\
		
		$k$ & Wear coefficients (also called Archard's constant) & / \\
		
		$T$ & The time since the steps were completed & $day$ \\
		
		$W$ & The width of the steps & $m$ \\
		
		$L$ & The length of the steps & $m$ \\
		
		$A_i$ & The area of the $i$-th step & $m^2$ \\
		
		$n$ & The total number of steps & / \\
		
		$\rho$ & The pedestrian density & $people/m^2$ \\
		
		$v$ & The velocity vector of pedestrians & $m/s$ \\
		
		$\nabla \cdot (\rho v)$ & The flux divergence & / \\
		
		$b$ & Experience factor & / \\
		
		% 底部粗线
		\noalign{\hrule height 1.5pt}
	\end{tabular}
	
	% 表格下方的注释(左对齐,斜体)
	\vspace{0.2cm} % 与表格的间距
	\begin{minipage}{0.9\textwidth} % 限制宽度以匹配表格视觉
		\textit{*There are some variables that are not listed here and will be discussed in detail below.}
	\end{minipage}
\end{table}




















\newpage
\section{Model Preparation}
\subsection{Data Processing}

\noindent We selected typical ancient architectural stairs as the study object. Computer vision was used to obtain and process images.
\begin{itemize}[label=\Large\textbullet,leftmargin=*,itemsep=0.6ex,topsep=0.5ex]
    \item \textbf{Step 1: Image processing}\\
    Original images were converted to grayscale and denoised via Gaussian filtering.
\end{itemize}

\begin{itemize}[label=\Large\textbullet,leftmargin=*,itemsep=0.6ex,topsep=0.5ex]
    \item \textbf{Step 2: Feature extraction}\\
    Edge detection and key feature extraction were performed on the processed images.
    \item \textbf{Step 3: Data processing}\\
    Structured datasets were formed for subsequent modeling and analysis.
\end{itemize}

\subsection{Solution of .....}



\newpage
\section{Model I:}
\setlength{\parindent}{2em} % 设置缩进为2个字符宽度


The core symbols and their definitions used in this study are summarized in Table~\ref{table:notations}, providing an overview of the key parameters and their related meanings.





\begin{table}[h]
	\centering
	\caption{table}
	\label{table}
	\begin{tabular}{lcr}
		\toprule
		A&B&C\\
		\midrule
		1&2&3\\
		4&5&6\\
		\bottomrule		
	\end{tabular}
\end{table}


The core symbols and their definitions used in this study are summarized in Table\ref{table}, providing an overview of the key parameters and their related meanings.

\begin{table}[h]
	\centering
	\caption{table}

	
	
	\begin{tabular}{
			>{\centering\arraybackslash}p{0.2\linewidth} >{\centering\arraybackslash}p{0.6\linewidth}
			>{\centering\arraybackslash}p{0.1\linewidth}
			} 
		
		
		\toprule 
		
		
		Symbol& Description&C\\ 
		
		
		\midrule 
		
		$\text{PSNR}$    & Peak signal-to-noise ratio for image quality evaluation&C \\
		

		
		\bottomrule
	\end{tabular}
\end{table}





\subsection{Proof of Lagrange Mean Value Theorem}
\subsubsection{Step 1: Construct the Auxiliary Function}
To apply Rolle's Theorem, we construct an auxiliary function $F(x)$, whose geometric meaning is "the difference between the ordinate of the curve $y=f(x)$ and the ordinate of the chord $AB$" (where $A(a,f(a))$ and $B(b,f(b))$ are the endpoints of the curve on $[a,b]$).

The equation of the straight line passing through $A$ and $B$ is:
\[
y - f(a) = \frac{f(b) - f(a)}{b - a}(x - a)
\]
which can be rewritten as:
\[
y = f(a) + \frac{f(b) - f(a)}{b - a}(x - a)
\]

Thus, the auxiliary function is defined as:
\[
F(x) = f(x) - \left[ f(a) + \frac{f(b) - f(a)}{b - a}(x - a) \right]
\]


To apply Rolle's Theorem, we construct an auxiliary function $F(x)$, whose geometric meaning is "the difference between


$$F(x)$$
\begin{equation}
	y = f(a) + \frac{f(b) - f(a)}{b - a}(x - a)
\end{equation}
