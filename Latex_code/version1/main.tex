\documentclass{mcmthesis}  % 当使用 CTeX 套装时请注释上一行使用该行的设置
\mcmsetup{tstyle=\color{red}\bfseries,%修改题号,队号的颜色和加粗显示,黑色可以修改为 black
tcn = 2611613, problem =A, %修改队号,参赛题号
sheet = true, titleinsheet = true, keywordsinsheet = true,
titlepage = false, abstract = true}
\usepackage{float}
%四款字体可以选择
\usepackage{times}
%\usepackage{newtxtext}
%\usepackage{palatino}
%\usepackage{txfonts}

\usepackage[utf8]{inputenc}
\usepackage{indentfirst}  %首行缩进,注释掉,首行就不再缩进。
\usepackage{lipsum}

\usepackage{ctex}  %这个包是确保可以用中文进行打草稿的

% 必备宏包
\usepackage{amsmath}  % 数学公式
\usepackage{amssymb}  % 数学符号
\usepackage[table]{xcolor} % 表格颜色
\usepackage{float}    %以此使用 [H] 固定表格位置
\usepackage{array}    % 调整表格行高
\usepackage{caption}  % 设置标题格式

\usepackage{amssymb} 
\usepackage{pifont} % 提供了大量精美图标
\usepackage{xcolor} % 甚至可以加颜色
\usepackage{enumitem}
\usepackage{tocloft}
\renewcommand{\cftbeforetoctitleskip}{0pt}
\renewcommand{\cftaftertoctitleskip}{0pt}
\renewcommand{\cfttoctitlefont}{\Large\bfseries}
\renewcommand{\contentsname}{Contents}
\renewcommand{\abstractname}{Abstract}
\renewcommand{\cftsecfont}{\bfseries}
\renewcommand{\cftsecpagefont}{\bfseries}
\renewcommand{\cftsubsecfont}{\normalfont}
\renewcommand{\cftsubsecpagefont}{\normalfont}
\setlength{\cftbeforesecskip}{2pt}
\setlength{\cftbeforesubsecskip}{0.5pt}
\setlength{\cftbeforesubsubsecskip}{0pt}
\setlength{\cftsecindent}{0pt}
\setlength{\cftsubsecindent}{1.5em}
\setlength{\cftsubsubsecindent}{3em}
\renewcommand{\cftdotsep}{2}
\renewcommand{\cftsecnumwidth}{2.5em}
\setcounter{tocdepth}{2}

%换行且有缩进用\\\indent
\title{our's title}
















\begin{document}


%(本文分析了为了研究,我们在进行分析后应用了模型。然后使用然后我们建立了一个来寻求之间关系的模型实际上,该模型可以用来分析本文对分析了模型的灵敏性,因此原始数据影响模型的稳定性,但是结果仍可以接受)

%(摘要中第一段对问题进行总体描述分析,之后对问题的建模、求解基本是1-2段写完的,然后单独一段对灵敏度等进行描述)

%(必须按要求写在特定的控制页面上,摘要的实际长度略多于半页(2/3页左右),提交时作为参赛论文的首页。

%摘要不能过长,在写完全文后再写摘要,摘要不能用文中的句子简单拼凑,要重新构思,反复推敲并修改到满意为止;摘要的第一句话尤其重要,可以适当运用问句;摘要中最好不要使用数学公式;通过阅读摘要,评审者能够全面了解作者的思路、方法以及主要结论)
%%%%%%%%%%%%%%%%%%%%%%%%%%%%%%%%%%%%%%%%%%%%%%%%%%%%%%%%%%%%%%%%%%%%%
\begin{abstract}

good
\textbf{good}
\underline{good}
\textsl{good}
\emph{good}

{\large good}
{\LARGE good}
{\huge good}
{\Huge good}

\begin{itemize}
	\item good
	\item good
	\begin{itemize}
		\item good
		\item good
		\begin{itemize}
			\item good
			\item good
			\begin{itemize}
				\item good
				\item good
			\end{itemize}
		\end{itemize}
	\end{itemize}
\end{itemize}




\begin{enumerate}
	\item good
	\item good
	\begin{enumerate}
		\item good
		\item good
		\begin{enumerate}
			\item good
			\item good
			\begin{enumerate}
				\item good
				\item good				
			\end{enumerate}
		\end{enumerate}
	\end{enumerate}
\end{enumerate}

$\blacksquare$ \\\indent
$\blacktriangleright$\\\indent
$\bullet$\\\indent
\begin{itemize}[label=\textcolor{blue}{$\blacktriangleright$}] % 蓝色的实心三角
	\item \textbf{Data Pre-processing:} Cleaning the raw data...
	\item \textbf{Model Construction:} Establishing the differential equations...
	\item \textbf{Result Validation:} Comparing with real-world datasets...
\end{itemize}



\begin{keywords}
SDCBJHDSBJCDSJBJCDBSJVBSDB;SDCHJSBVSD;SDCDSC;SCSDCDSCS
\end{keywords}


\end{abstract}

\maketitle

\makeatletter
\renewcommand\tableofcontents{%
    \centerline{\normalfont\Large\bfseries\sffamily\contentsname
        \@mkboth{%
           \MakeUppercase\contentsname}{\MakeUppercase\contentsname}}%
    \vskip 1.5ex%
    \@starttoc{toc}%
    }
\makeatother
\tableofcontents


\newpage
\section{Introduction}


\subsection{Problem Background}







\subsection{Restatement of the Problem}

\noindent Considering the background information and restricted conditions identified in the problem statement, we need to solve the following problems:
\begin{itemize}[label=\Large\textbullet,leftmargin=*,itemsep=0.6ex,topsep=0.5ex]
    \item Develop a model to describe the frequency of stair usage.
    \item Create a model to determine whether a certain direction of travel was favored.
\end{itemize}

\begin{itemize}[label=\Large\textbullet,leftmargin=*,itemsep=0.6ex,topsep=0.5ex]
    \item Develop a model to determine whether people traveled single file or side by side.
    \item Build a model to evaluate whether the observed wear matches historical and environmental factors.
    \item Create a model to estimate the age of the stairwell and its reliability.
    \item Design a model to identify signs of repairs or renovations.
    \item Determine whether the wear is consistent with materials from specific sources.
    \item Estimate typical usage and identify whether heavy short-term or prolonged lighter usage fits the observed patterns.
\end{itemize}







\subsection{Our Work}

\noindent In this paper, we carried out the following work to study stair usage frequency, service life, current condition, and pedestrian behavior patterns when using the stairs.
\begin{itemize}[label=\Large\ding{226},leftmargin=*,itemsep=0.6ex,topsep=0.5ex]
    \item \textbf{Task 1: Model Preparation}\\
    We determined material and environmental conditions of the target stairs, applied computer vision to obtain and process data, and fitted the wear depth distribution.
    
    \item \textbf{Task 2: Stair Wear Analysis Model}\\
    We modeled pedestrian traffic to characterize usage frequency, inferred preferred walking direction via changes in wear distribution, and computed pedestrian density from traffic flow to judge single-file or side-by-side walking.
    
    \item \textbf{Task 3:}\\
    \begin{itemize}[label=\Large\textbullet,leftmargin=*,itemsep=0.6ex,topsep=0.5ex]
        \item Analyze the relation among stair age, wear distribution, and pedestrian traffic.
        \item Compute theoretical wear from traffic and compare with measured wear; assess error.
        \item Estimate stair age from wear–traffic relationships and evaluate reliability.
        \item Identify signs of repairs or renovations via coefficient changes and verification.
        \item Check material consistency with presumed sources and verify with features.
        \item Estimate typical usage; distinguish heavy short-term vs. prolonged lighter usage.
    \end{itemize}
\end{itemize}
\section{Assumptions and Justifications}

\begin{itemize}[label=\Large\textbullet,leftmargin=*,itemsep=0.6ex,topsep=0.5ex]
    \item \textbf{Assumption 1: 示例标题}
    
    \textbf{Justification:} 示例文本。
    
    \item \textbf{Assumption 2: 示例标题}
    
    \textbf{Justification:} 示例文本。
    
    \item \textbf{Assumption 3: 示例标题}
    
    \textbf{Justification:} 示例文本。
    
    \item \textbf{Assumption 4: 示例标题}
    
    \textbf{Justification:} 示例文本。
\end{itemize}







\newpage
\section{Notations}
	The key mathematical notations used in this paper are listed in Table \ref{tab:notations}.

% 表格开始
\begin{table}[H]
	\centering
	% 设置标题格式:加粗
	\captionsetup{font=bf} 
	\caption{Notations used in this paper}
	\label{tab:notations}
	
	% 增加行高,使表格看起来更舒展(类似图片效果)
	\renewcommand{\arraystretch}{1.4}
	
	% 定义列格式:|c|c|c| 表示三列居中,且有竖线分隔
	\begin{tabular}{|c|c|c|}
		% 顶部粗线 (自定义粗细为1.5pt)
		\noalign{\hrule height 1.5pt}
		
		% 表头行:灰色背景,加粗斜体文字
		\rowcolor{gray!30} 
		\textit{\textbf{Symbols}} & \textit{\textbf{Definition}} & \textit{\textbf{Units}} \\
		
		\hline
		
		% 内容行
		$d(x,y)$ & The wear depth at each location $(x, y)$ & $m$ \\
		
		$N$ & Pedestrian flow & $people/s$ \\
		
		$k$ & Wear coefficients (also called Archard's constant) & / \\
		
		$T$ & The time since the steps were completed & $day$ \\
		
		$W$ & The width of the steps & $m$ \\
		
		$L$ & The length of the steps & $m$ \\
		
		$A_i$ & The area of the $i$-th step & $m^2$ \\
		
		$n$ & The total number of steps & / \\
		
		$\rho$ & The pedestrian density & $people/m^2$ \\
		
		$v$ & The velocity vector of pedestrians & $m/s$ \\
		
		$\nabla \cdot (\rho v)$ & The flux divergence & / \\
		
		$b$ & Experience factor & / \\
		
		% 底部粗线
		\noalign{\hrule height 1.5pt}
	\end{tabular}
	
	% 表格下方的注释(左对齐,斜体)
	\vspace{0.2cm} % 与表格的间距
	\begin{minipage}{0.9\textwidth} % 限制宽度以匹配表格视觉
		\textit{*There are some variables that are not listed here and will be discussed in detail below.}
	\end{minipage}
\end{table}




















\newpage
\section{Model Preparation}
\subsection{Data Processing}

\noindent We selected typical ancient architectural stairs as the study object. Computer vision was used to obtain and process images.
\begin{itemize}[label=\Large\textbullet,leftmargin=*,itemsep=0.6ex,topsep=0.5ex]
    \item \textbf{Step 1: Image processing}\\
    Original images were converted to grayscale and denoised via Gaussian filtering.
\end{itemize}

\begin{itemize}[label=\Large\textbullet,leftmargin=*,itemsep=0.6ex,topsep=0.5ex]
    \item \textbf{Step 2: Feature extraction}\\
    Edge detection and key feature extraction were performed on the processed images.
    \item \textbf{Step 3: Data processing}\\
    Structured datasets were formed for subsequent modeling and analysis.
\end{itemize}

\subsection{Solution of .....}



\newpage
\section{Model I:}
\setlength{\parindent}{2em} % 设置缩进为2个字符宽度


The core symbols and their definitions used in this study are summarized in Table~\ref{table:notations}, providing an overview of the key parameters and their related meanings.





\begin{table}[h]
	\centering
	\caption{table}
	\label{table}
	\begin{tabular}{lcr}
		\toprule
		A&B&C\\
		\midrule
		1&2&3\\
		4&5&6\\
		\bottomrule		
	\end{tabular}
\end{table}


The core symbols and their definitions used in this study are summarized in Table\ref{table}, providing an overview of the key parameters and their related meanings.

\begin{table}[h]
	\centering
	\caption{table}

	
	
	\begin{tabular}{
			>{\centering\arraybackslash}p{0.2\linewidth} >{\centering\arraybackslash}p{0.6\linewidth}
			>{\centering\arraybackslash}p{0.1\linewidth}
			} 
		
		
		\toprule 
		
		
		Symbol& Description&C\\ 
		
		
		\midrule 
		
		$\text{PSNR}$    & Peak signal-to-noise ratio for image quality evaluation&C \\
		

		
		\bottomrule
	\end{tabular}
\end{table}





\subsection{Proof of Lagrange Mean Value Theorem}
\subsubsection{Step 1: Construct the Auxiliary Function}
To apply Rolle's Theorem, we construct an auxiliary function $F(x)$, whose geometric meaning is "the difference between the ordinate of the curve $y=f(x)$ and the ordinate of the chord $AB$" (where $A(a,f(a))$ and $B(b,f(b))$ are the endpoints of the curve on $[a,b]$).

The equation of the straight line passing through $A$ and $B$ is:
\[
y - f(a) = \frac{f(b) - f(a)}{b - a}(x - a)
\]
which can be rewritten as:
\[
y = f(a) + \frac{f(b) - f(a)}{b - a}(x - a)
\]

Thus, the auxiliary function is defined as:
\[
F(x) = f(x) - \left[ f(a) + \frac{f(b) - f(a)}{b - a}(x - a) \right]
\]


To apply Rolle's Theorem, we construct an auxiliary function $F(x)$, whose geometric meaning is "the difference between


$$F(x)$$
\begin{equation}
	y = f(a) + \frac{f(b) - f(a)}{b - a}(x - a)
\end{equation}



\newpage
\section{Model II}










\subsection{Modeling and Solving of Problem 1}
\subsubsection{Problem Analysis}



%本问是一个综合性数据分析与预测类问题,涉及特征相关性分析、风险聚类评估与概率预测建模。为系统研究洪水发生的关键影响因素及风险预警机制,本节设计了一个基于多指标融合的洪水风险分析与预测模型。该模型通过相关性分析、聚类分析、权重计算与机器学习预测四个阶段,对 20 个气象与地理指标进行系统建模与验证,旨在建立科学的洪水风险预警体系。
%
%首先,获取附件 train.csv 中的历史洪水监测数据,涵盖 20 个气象与环境指标以及洪水发生概率。利用 Python 的 Pandas 和 NumPy 库对数据进行清洗、缺失值补全与标准化处理,确保各特征处于统一量纲。
%
%接着,采用皮尔逊相关系数(Pearson Correlation)方法,分析各指标与洪水发生概率之间的线性与综合关系。
%
%在此基础上,利用K-means 聚类算法对洪水发生概率进行分群,将样本划分为高风险、中风险与低风险三类。
%
%随后,为构建量化的风险评估体系,采用熵权法指标加权方法,计算各指标在洪水风险评估中的相对重要性,形成洪水风险预警评价模型。
%
%最后,基于筛选出的关键指标,构建洪水发生概率预测模型。选用梯度提升树(XGBoost)机器学习算法,利用训练集与测试集划分、交叉验证与特征重要性分析评估模型性能。
%
%该综合模型通过“指标分析—聚类识别—权重计算—预测验证”的流程,实现了从数据特征识别到概率预测的完整闭环,为洪水风险监测与预警提供了可行的科学依据。整体研究流程如图 X 所示。
%











\subsubsection{Model Preparation}


%数据预处理工作+模型选择工作

%缺失值-异常值-重复值-标准化-编码-特征工程【一般情况写在模型建立里面】

%写每一个步骤的时候,需要按照:
%表明处理理由
%+介绍处理方法【一句话引入理论+公式展示具体计算方法+公式的解释语句】
%+处理前后的对比图
%+结果总结文字,主要就是描述图
























\subsubsection{Model Construction}

%
%三明治结构:引入文字+具体步骤+总结文字



















\subsubsection{Model Solution}




\subsection{Modeling and Solving of Problem 1}
\subsubsection{Problem Analysis}

\subsubsection{Model Preparation}

\subsubsection{Model Construction}

\subsubsection{Model Solution}





\input{2-3.part-2-3}








\newpage

\section{Sensitivity Analysis}
敏感性分析内容





\newpage
\section{Model Evaluation and Further Discussion}
\subsection{Model Evaluation}

%6-1-1
%%%%%%%%%%%%%%%%%%%%%%%%%%%%%%%%%%%%%%%%%%%%%%%%%%%%%%%%%%%%%%%%%%%%%%%%%
\subsubsection{Advantages}

\begin{itemize}[label=\Large\ding{117},leftmargin=*,itemsep=0.6ex,topsep=0.5ex]
    \item 双变量分布刻画磨损形态并匹配参数差异。
    \item 采用粒子群等优化提升拟合效率与精度。
    \item 将人群流量抽象为流体并用数值方法求解。
    \item 引入脉冲函数模拟短时大量人流场景。
    \item 模型方法成熟、准确性较高、可复用性强。
\end{itemize}




%6-1-2
%%%%%%%%%%%%%%%%%%%%%%%%%%%%%%%%%%%%%%%%%%%%%%%%%%%%%%%%%%%%%%%%%%%%%%%%%
\subsubsection{Limitations}

\begin{itemize}[label=\Large\ding{117},leftmargin=*,itemsep=0.6ex,topsep=0.5ex]
    \item 现场测量手段有限,数据精度受约束。
    \item 样本场景单一,外推到多类楼梯存在偏差。
\end{itemize}


%6-2
%%%%%%%%%%%%%%%%%%%%%%%%%%%%%%%%%%%%%%%%%%%%%%%%%%%%%%%%%%%%%%%%%%%%%%%%%
\subsection{Future Work}

%6-2-1
%%%%%%%%%%%%%%%%%%%%%%%%%%%%%%%%%%%%%%%%%%%%%%%%%%%%%%%%%%%%%%%%%%%%%%%%%
\subsubsection{Model extension}


%6-2-2
%%%%%%%%%%%%%%%%%%%%%%%%%%%%%%%%%%%%%%%%%%%%%%%%%%%%%%%%%%%%%%%%%%%%%%%%%
\subsubsection{Model application}










\section{Conclusions}














\cite{Krawitz2025}


\newpage

\addcontentsline{toc}{section}{Reference}
\renewcommand{\refname}{Reference}  % 英文环境用\refname
\begin{thebibliography}{99}
	

	
	\bibitem{Bundela2024}
	Bundela B, Sharma S, Singh B R.
	A Review Article on Relation between Mathematical Modelling and Machine Learning[J].
	Journal of China University of Mining \& Technology, 2024, 29(2): 123–129.
	
	\bibitem{Krawitz2025}
	Krawitz J, Schukajlow S, Yang X, et al. 
	A Systematic Review of International Perspectives on Mathematical Modelling: Modelling Goals and Task Characteristics[J]. 
	ZDM – Mathematics Education, 2025, 57: 193–212.
	
	
	
	
	
	\bibitem{Lyon2020}
	Lyon J A, Magana A J.
	A Review of Mathematical Modeling in Engineering Education[J].
	International Journal of Engineering Education, 2020, 36(1A): 101–116.
	
	\bibitem{Huang2022}
	Huang S, Feng W, Tang C, et al.
	Partial Differential Equations Meet Deep Neural Networks: A Survey[EB/OL].
	arXiv:2211.05567, 2022.
	
	\bibitem{Zino2021}
	Zino L, Cao M.
	Analysis, Prediction, and Control of Epidemics: A Survey from Scalar to Dynamic Network Models[EB/OL].
	arXiv:2103.00181, 2021.
	
	\bibitem{Turkey2023}
	Author Unknown.
	Review of Mathematical Modeling Research: A Descriptive Content Analysis[J].
	Balıkesir NEF Journal, 2023.
	
	\bibitem{SoftRobotics2023}
	Author Unknown.
	Mathematical modelling, analysis and control in soft robotics: a survey[J].
	Journal of the Brazilian Society of Mechanical Sciences and Engineering, 2023.
	
	\bibitem{PDE2012}
	Author Unknown.
	Mathematical Modelling at a Glance: A Theoretical Study[J].
	Procedia - Social and Behavioral Sciences, 2012, 46: 3980–3984.
	
	\bibitem{Surrogate2024}
	Author Unknown.
	A Review of Recent Advances in Surrogate Models for Uncertainty Quantification[J].
	Computer Methods in Applied Mechanics and Engineering, 2024.
	
\end{thebibliography}

































































\newpage
\section*{Appendices}
\addcontentsline{toc}{section}{Appendices}



\newpage
\AImatter
\begin{ReportAiUse}{9}



\bibitem{AI1}

Bing AI\\   %这里写AI的类型
Query1: \\ %这里写你的问题
Output:     %这里写ai的回复





\bibitem{AI1}

Bing AI\\   %这里写AI的类型
Query1: \\ %这里写你的问题
Output:     %这里写ai的回复





\bibitem{AI1}

Bing AI\\   %这里写AI的类型
Query1: \\ %这里写你的问题
Output:     %这里写ai的回复




\bibitem{AI1}

Bing AI\\   %这里写AI的类型
Query1: \\ %这里写你的问题
Output:     %这里写ai的回复









\end{ReportAiUse}





\end{document}
