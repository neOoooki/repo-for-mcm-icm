\subsection{Modeling and Solving of Problem 1}
\subsubsection{Problem Analysis}



%本问是一个综合性数据分析与预测类问题,涉及特征相关性分析、风险聚类评估与概率预测建模。为系统研究洪水发生的关键影响因素及风险预警机制,本节设计了一个基于多指标融合的洪水风险分析与预测模型。该模型通过相关性分析、聚类分析、权重计算与机器学习预测四个阶段,对 20 个气象与地理指标进行系统建模与验证,旨在建立科学的洪水风险预警体系。
%
%首先,获取附件 train.csv 中的历史洪水监测数据,涵盖 20 个气象与环境指标以及洪水发生概率。利用 Python 的 Pandas 和 NumPy 库对数据进行清洗、缺失值补全与标准化处理,确保各特征处于统一量纲。
%
%接着,采用皮尔逊相关系数(Pearson Correlation)方法,分析各指标与洪水发生概率之间的线性与综合关系。
%
%在此基础上,利用K-means 聚类算法对洪水发生概率进行分群,将样本划分为高风险、中风险与低风险三类。
%
%随后,为构建量化的风险评估体系,采用熵权法指标加权方法,计算各指标在洪水风险评估中的相对重要性,形成洪水风险预警评价模型。
%
%最后,基于筛选出的关键指标,构建洪水发生概率预测模型。选用梯度提升树(XGBoost)机器学习算法,利用训练集与测试集划分、交叉验证与特征重要性分析评估模型性能。
%
%该综合模型通过“指标分析—聚类识别—权重计算—预测验证”的流程,实现了从数据特征识别到概率预测的完整闭环,为洪水风险监测与预警提供了可行的科学依据。整体研究流程如图 X 所示。
%











\subsubsection{Model Preparation}


%数据预处理工作+模型选择工作

%缺失值-异常值-重复值-标准化-编码-特征工程【一般情况写在模型建立里面】

%写每一个步骤的时候,需要按照:
%表明处理理由
%+介绍处理方法【一句话引入理论+公式展示具体计算方法+公式的解释语句】
%+处理前后的对比图
%+结果总结文字,主要就是描述图
























\subsubsection{Model Construction}

%
%三明治结构:引入文字+具体步骤+总结文字



















\subsubsection{Model Solution}


