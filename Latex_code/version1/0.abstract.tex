%(本文分析了为了研究,我们在进行分析后应用了模型。然后使用然后我们建立了一个来寻求之间关系的模型实际上,该模型可以用来分析本文对分析了模型的灵敏性,因此原始数据影响模型的稳定性,但是结果仍可以接受)

%(摘要中第一段对问题进行总体描述分析,之后对问题的建模、求解基本是1-2段写完的,然后单独一段对灵敏度等进行描述)

%(必须按要求写在特定的控制页面上,摘要的实际长度略多于半页(2/3页左右),提交时作为参赛论文的首页。

%摘要不能过长,在写完全文后再写摘要,摘要不能用文中的句子简单拼凑,要重新构思,反复推敲并修改到满意为止;摘要的第一句话尤其重要,可以适当运用问句;摘要中最好不要使用数学公式;通过阅读摘要,评审者能够全面了解作者的思路、方法以及主要结论)
%%%%%%%%%%%%%%%%%%%%%%%%%%%%%%%%%%%%%%%%%%%%%%%%%%%%%%%%%%%%%%%%%%%%%
\begin{abstract}

good
\textbf{good}
\underline{good}
\textsl{good}
\emph{good}

{\large good}
{\LARGE good}
{\huge good}
{\Huge good}

\begin{itemize}
	\item good
	\item good
	\begin{itemize}
		\item good
		\item good
		\begin{itemize}
			\item good
			\item good
			\begin{itemize}
				\item good
				\item good
			\end{itemize}
		\end{itemize}
	\end{itemize}
\end{itemize}




\begin{enumerate}
	\item good
	\item good
	\begin{enumerate}
		\item good
		\item good
		\begin{enumerate}
			\item good
			\item good
			\begin{enumerate}
				\item good
				\item good				
			\end{enumerate}
		\end{enumerate}
	\end{enumerate}
\end{enumerate}

$\blacksquare$ \\\indent
$\blacktriangleright$\\\indent
$\bullet$\\\indent
\begin{itemize}[label=\textcolor{blue}{$\blacktriangleright$}] % 蓝色的实心三角
	\item \textbf{Data Pre-processing:} Cleaning the raw data...
	\item \textbf{Model Construction:} Establishing the differential equations...
	\item \textbf{Result Validation:} Comparing with real-world datasets...
\end{itemize}



\begin{keywords}
SDCBJHDSBJCDSJBJCDBSJVBSDB;SDCHJSBVSD;SDCDSC;SCSDCDSCS
\end{keywords}


\end{abstract}